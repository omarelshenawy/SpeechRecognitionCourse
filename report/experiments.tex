The original 61 phonemes in the TIMIT corpus was mapped into 39 phonemes as suggested in {mapping}. This mapping is shown in table \ref{table:mapping}. The 39 phonemes are then used to both train and evaluate the model. 
\begin{center}
\begin{table}[H]
    \begin{tabular}{ | l | l |}
    \hline
    aa, ao & aa \\ \hline
    ah, ax, ax-h  & ah \\ \hline
    er, axr & er \\ \hline
    hh, hv & hh \\ \hline   
    ih, ix & ih \\ \hline
	l, el & l \\ \hline
	m, em & m \\ \hline
	n, en, nx & n \\ \hline
	ng, eng & ng \\ \hline
	sh, zh & sh \\ \hline
	uw, ux & uw \\ \hline
	pcl, tcl, kcl, bcl, dcl, gcl, h\#, pau, epi & sil \\ \hline
	q & - \\ \hline
    \hline
    \end{tabular}
    \caption{Mapping the 61 original TIMIT phonemes(left) into 39 phonemes(right) as suggested in \cite{mapping}}
    \label{table:mapping}
\end{table}
\end{center}


we tried different networks 
different data sets (train/validation split)
results, including training time
in a table